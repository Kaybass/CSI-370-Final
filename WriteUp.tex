% Alexander Bean Apmann
%
% CSI-370
%
% Professor Brian Hall
%
% March 4, 2015


\documentclass[11pt,letterpaper]{article}

% All the packages I used came with the default installation of texlive
\usepackage{ifpdf}
\usepackage[pdftex]{graphicx}
\usepackage{mla}
\usepackage{listings}
\usepackage{hyperref}
\usepackage{float}
\begin{document}
\lstset{language=[x86masm]Assembler}
\begin{mla}{Alexander Bean}{Apmann}{Professor Hall}{CSI-370}{\today}{Final Project Writeup}

For my project I decided to redo a program I wrote in CSI-281 that compared the times of different sorting algorithms. What I changed in the project was that instead of just observing the speed of array sorting algorithms I wrote in assembly I would compare them to their C counterparts. The sorts I used were Bubble Sort, Selection Sort, Quicksort, and Insertion Sort. I wanted to write some sort of algorithm in assembly for my final project and this project seemed like a good balance between being simple enough to implement but complex enough to be worth merit.

\begin{figure}[h!]
	\caption{Reasons this was a good project idea}
	\centering
		\includegraphics[scale=.35]{graph.png}
\end{figure}

I decided to write the sorts in assembly and the rest of the program in C. This made sense to me for a couple of reasons. First it would be easier to link the sorts written in assembly to a program in C than the other way around as I already know how to do that. Second the project is about the sorts, not the script that tests the sorts. Third while it wouldn't be too hard to get the number of clock cycles it takes to do a group of instructions or generate a set of random numbers in assembly, time.h and rand() are guaranteed to work and I wouldn't want me writing a bad function that does either of those to mess up my results.

\begin{figure}[h!]
	\caption{Rough linking of C and ASM Selection Sort}
	\centering
		\includegraphics[scale=.35]{sortcomp.png}
\end{figure}

To write the sorts I used sorts in C as a reference (except in the case of bubble sort where I just translated the midterm program from GAS to NASM syntax). To debug the sorts as I was writing them I used the SASM assembly IDE to debug the sort as I was writing it. Instead of linking the sort as I was writing it to a C program I made the sort a stand alone program and in the data section put an array and put the length of the array as an integer. Once I confirmed that my implementation of the sort was correct I converted the a function and linked it to the C code.

The C program itself is pretty straightforward. It generates 4 pairs of arrays with 10000 random numbers in them, times the sorts with these arrays and then outputs the data. The main file also contains the sorts I used in C. These sorts were taken from http://rosettacode.org/. This was not to be lazy, but because these sorts would lose their purpose as a control against my sorts in assembly if they were written by me, the same reason I am using time.h and stdlid's rand(). I built the project by assembling the sorts individually with nasm then building the C program with GCC and then linking the object files together with GCC.

The results of my project were super awesome. Every one of my sorts were faster than their C counterpart except Bubble Sort. I do not know exactly why this is but I suspect that it is that I used a register as a temporary spot while switching values instead of another spot in memory like the C sorts do. The times I used in the graph below are from one run of the program, however they are approximately what I got every time I ran the program. Even though three out of four is a C grade I think it is a pretty great outcome. I think that if I were to look into optimizing sorts with GCC I might make the C sorts faster than mine but I am happy I was able to make my sorts faster than the unoptimized GCC ones.

\begin{figure}[h!]
	\caption{Results of my project}
	\centering
		\includegraphics[scale=.7]{graph2.png}
\end{figure}


\end{mla}
\end{document}
